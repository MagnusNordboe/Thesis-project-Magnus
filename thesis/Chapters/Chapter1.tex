% Chapter 1

\chapter{Introduction} % Main chapter title

\label{Chapter1} % For referencing the chapter elsewhere, use \ref{Chapter1} 

%----------------------------------------------------------------------------------------

% Define some commands to keep the formatting separated from the content 
\newcommand{\keyword}[1]{\textbf{#1}}
\newcommand{\tabhead}[1]{\textbf{#1}}
\newcommand{\code}[1]{\texttt{#1}}
\newcommand{\file}[1]{\texttt{\bfseries#1}}
\newcommand{\option}[1]{\texttt{\itshape#1}}

%----------------------------------------------------------------------------------------

\section{Keywords}

Observability \\
Microservices \\
Preprocessing \\
Visualization \\


\section{Research area and questions}
The research area for this thesis is microservices and machine learning. An inevitable intersection appears between the two fields
when the need for analysis of data from microservice systems arises. This is because these complex systems are capable of generating a
vast amount of metric data about themselves. The amount of data is too large for a human to read, comprehend and reason about in any efficient capacity.
This can make it very difficult to draw useful conclusions about the system. This is where machine learning comes in as a natural answer to the problem:
While humans are limited in the amount of information they can comprehend, machines typically only benefit from absurdly large datasets to draw from, as long as
the data is of acceptable quality. However, the field of machine learning applications on microservice system data is still immature as of time of writing.
This thesis seeks to analyze a small subset of this field. The scope will be contained to creating a framework for creating, collecting, aggregating and preprocessing data driven model data about a microservice system, using only centralized logging.
The underlying idea is that one might be able to model the behavior of the system under heavy load beforehand, by using a stress testing tool and aggregating and preprocessing the data and understanding the system's behavior before it happens.
\keyword{Research question 1}: Using stress testing, centralized logging and preprocessing, is it possible to gain meaningful information about the system's bottlenecks before they happen in production?
\keyword{Research question 2}: What are the current main challenges to effectively utilize metric data from microservice systems?

\section{Approach}
%Researched ways of generating good data about microservice systems to use
%Used microservices-demo and its limited collection
%Purpose became to find what one could from the limited data that was collected.
To research these questions and try to create meaningful results that can have a real impact, I formulated a hypothesis to work as the base for the project.
To test this hypothesis, a fork of the open source demo microservice project "sockshop" by Weaveworks was used \cite*{Microservices-demo-sockshop}. This system was hosted on a local machine and stress tested using various configurations.
Locust, an API stress testing tool. The full spectrum of available Prometheus metric data was collected in 601 time series features into CSV files and labeled. There were a total of five classes, representing different endpoints connecting to different underlying microservices.
This data was then analyzed and preprocessed to reduce noise and extract important features.
Several methods of preprocessing and classifying were quantitatively compared. Finally, the results of the analysis were discussed and conclusions drawn.

\section{List of important terms}
\begin{itemize}
	\item \keyword{Multivariate Time Series (MTS)}: Time series data with two or more variables.
	\item \keyword{Instance}: All the data collected from a system in a specific time frame, represented as an MTS.
	\item \keyword{Feature}: A unique variable in the time series that expresses some kind of information about the system. Collected at fixed intervals to form an MTS.
	\item \keyword{Time frame}: The period of time recorded in a time series. Each instance in this thesis corresponds to a specific time frame.
	\item \keyword{Centralized logging}: Umbrella term for various microservice info logging methods that collect logs from individual microservices and stores them together.
	\item \keyword{Distributed tracing}: A more in depth, but also more resource intensive method for gathering information about a microservice system's runtime behavior, where it tracks the entire path of invocations through the system.
\end{itemize}

