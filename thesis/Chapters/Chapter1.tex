% Chapter 1

\chapter{Introduction} % Main chapter title

\label{Chapter1} % For referencing the chapter elsewhere, use \ref{Chapter1} 

%----------------------------------------------------------------------------------------

% Define some commands to keep the formatting separated from the content 
\newcommand{\keyword}[1]{\textbf{#1}}
\newcommand{\tabhead}[1]{\textbf{#1}}
\newcommand{\code}[1]{\texttt{#1}}
\newcommand{\file}[1]{\texttt{\bfseries#1}}
\newcommand{\option}[1]{\texttt{\itshape#1}}

%----------------------------------------------------------------------------------------

\section{Keywords}

Observability \\
Microservices \\
Signal-to-noise ration \\

\section{Abstract}
This thesis project aims to explore the practicality and efficacy of multivariate time series classification to quickly diagnose performance bottlenecks in a microservice system.
A simple microservice system running in Docker was forked from another project and minimally adjusted. Metrics were collected about the system using the Prometheus monitoring software, and stress testing was done at the API level with Pumba. 
The resulting data was fed into different MTS machine learning algorithms to judge their accuracy in predicting the source of the stress.
Finally, the results of these predictions were compared and the methods for collecting, treating and training on the data were documented to benefit future implementations of a similar approach.


\section{Research area and questions}
The research area for this thesis is microservices and machine learning. An inevitable intersection appears between the two fields
when the need for analysis of data from microservice systems arises. This is because these complex systems are capable of generating a 
vast amount of metric data about themselves. The amount of data is too large for a human to read, comprehend and reason about in any efficient capacity. 
This can make it very difficult to draw useful conclusions about the system. This is where machine learning comes in as a natural answer to the problem:
While humans are limited in the amount of information they can comprehend, machines typically only benefit from absurdly large datasets to draw from, as long as
the data is of acceptable quality. However, the field of machine learning applications on microservice system data is still immature as of time of writing. 
This thesis seeks to analyze a small subset of this field. The scope will be contained to identifying latency issues resulting from disproportionally high load in a part of the system, using only data from centralized logging.\\
\keyword{Research question 1}: How good are current machine learning classification algorithms at identifying causes for anomalous behavior in microservice systems?\\
\keyword{Research question 2}: What are the current main challenges to effectively utilize metric data from microservice systems?

\section{Approach}
%Researched ways of generating good data about microservice systems to use
%Used microservices-demo and its limited collection
%Purpose became to find what one could from the limited data that was collected.
To research these questions and try to create meaningful results that can have a real impact, I formulated a hypothesis to work as the base for the project. 
\textbf{\textit{It is possible to use stress testing software and centralized logging to create a machine learning model of the system behavior to classify and identify sources of latency in a microservice system.}}
To test this hypothesis, a fork of the open source demo microservice project "sockshop" by Weaveworks was used \cite*{Microservices-demo-sockshop}. This system was hosted on a local machine and stress tested using various configurations.
Locust, an API stress testing tool. The full spectrum of available Prometheus metric data was collected in 601 time series features into csv files and labeled. There were a total of five classes, representing different endpoints connecting to different underlying microservices.
This data was then analyzed both manually and mathematically, and preprocessed to reduce noise and extract important features. 
Several methods of preprocessing and classifying were quantitatively compared. Finally, the results of the analysis were discussed and conclusions drawn.

\section{List of important terms}
\begin{itemize}
\item \keyword{Dynamic Time Warping}: A method of statistically comparing time series across time points to judge similarity.
\item \keyword{Multivariate Time Series (MTS)}: Time series data with two or more variables.
\item \keyword{Instance}: All the data collected from a system in a specific time frame, represented as an MTS.
\item \keyword{Feature}: A unique variable in the time series that expresses some kind of information about the system. Collected at fixed intervals to form an MTS.
\item \keyword{Time frame}: The period of time recorded in a time series. Each instance in this thesis corresponds to a specific time frame.
\item \keyword{Centralized logging}: Umbrella term for various microservice info logging methods that collect logs from individual microservices and stores them together. 
\item \keyword{Distributed tracing}: 
\end{itemize}

\keyword{Appendices} -- this is the folder where you put the appendices. Each appendix should go into its own separate \file{.tex} file. An example and template are included in the directory.

% \keyword{Chapters} -- this is the folder where you put the thesis chapters. A thesis usually has about six chapters, though there is no hard rule on this. Each chapter should go in its own separate \file{.tex} file and they can be split as:
% \begin{itemize}
% \item Chapter 1: Introduction to the thesis topic
% \item Chapter 2: Background information and theory
% \item Chapter 3: (Laboratory) experimental setup
% \item Chapter 4: Details of experiment 1
% \item Chapter 5: Details of experiment 2
% \item Chapter 6: Discussion of the experimental results
% \item Chapter 7: Conclusion and future directions
% \end{itemize}
% This chapter layout is specialised for the experimental sciences, your discipline may be different.

% \keyword{Figures} -- this folder contains all figures for the thesis. These are the final images that will go into the thesis document.


% \section{Thesis Features and Conventions}\label{ThesisConventions}

% To get the best out of this template, there are a few conventions that you may want to follow.

% One of the most important (and most difficult) things to keep track of in such a long document as a thesis is consistency. Using certain conventions and ways of doing things (such as using a Todo list) makes the job easier. Of course, all of these are optional and you can adopt your own method.

% \subsection{Printing Format}

% This thesis template is designed for double sided printing (i.e. content on the front and back of pages) as most theses are printed and bound this way. Switching to one sided printing is as simple as uncommenting the \option{oneside} option of the \code{documentclass} command at the top of the \file{main.tex} file. You may then wish to adjust the margins to suit specifications from your institution.

% The headers for the pages contain the page number on the outer side (so it is easy to flick through to the page you want) and the chapter name on the inner side.

% The text is set to 11 point by default with single line spacing, again, you can tune the text size and spacing should you want or need to using the options at the very start of \file{main.tex}. The spacing can be changed similarly by replacing the \option{singlespacing} with \option{onehalfspacing} or \option{doublespacing}.

% \subsection{Tables}

% Tables are an important way of displaying your results, below is an example table which was generated with this code:

% {\small
% \begin{verbatim}
% \begin{table}
% \caption{The effects of treatments X and Y on the four groups studied.}
% \label{tab:treatments}
% \centering
% \begin{tabular}{l l l}
% \toprule
% \tabhead{Groups} & \tabhead{Treatment X} & \tabhead{Treatment Y} \\
% \midrule
% 1 & 0.2 & 0.8\\
% 2 & 0.17 & 0.7\\
% 3 & 0.24 & 0.75\\
% 4 & 0.68 & 0.3\\
% \bottomrule\\
% \end{tabular}
% \end{table}
% \end{verbatim}
% }

% \begin{table}
% \caption{The effects of treatments X and Y on the four groups studied.}
% \label{tab:treatments}
% \centering
% \begin{tabular}{l l l}
% \toprule
% \tabhead{Groups} & \tabhead{Treatment X} & \tabhead{Treatment Y} \\
% \midrule
% 1 & 0.2 & 0.8\\
% 2 & 0.17 & 0.7\\
% 3 & 0.24 & 0.75\\
% 4 & 0.68 & 0.3\\
% \bottomrule\\
% \end{tabular}
% \end{table}

% You can reference tables with \verb|\ref{<label>}| where the label is defined within the table environment. See \file{Chapter1.tex} for an example of the label and citation (e.g. Table~\ref{tab:treatments}).

% \subsection{Figures}

% There will hopefully be many figures in your thesis (that should be placed in the \emph{Figures} folder). The way to insert figures into your thesis is to use a code template like this:
% \begin{verbatim}
% \begin{figure}
% \centering
% \includegraphics{Figures/Electron}
% \decoRule
% \caption[An Electron]{An electron (artist's impression).}
% \label{fig:Electron}
% \end{figure}
% \end{verbatim}
% Also look in the source file. Putting this code into the source file produces the picture of the electron that you can see in the figure below.

% \begin{figure}[th]
% \centering
% \includegraphics{Figures/Electron}
% \decoRule
% \caption[An Electron]{An electron (artist's impression).}
% \label{fig:Electron}
% \end{figure}

% Sometimes figures don't always appear where you write them in the source. The placement depends on how much space there is on the page for the figure. Sometimes there is not enough room to fit a figure directly where it should go (in relation to the text) and so \LaTeX{} puts it at the top of the next page. Positioning figures is the job of \LaTeX{} and so you should only worry about making them look good!

% Figures usually should have captions just in case you need to refer to them (such as in Figure~\ref{fig:Electron}). The \verb|\caption| command contains two parts, the first part, inside the square brackets is the title that will appear in the \emph{List of Figures}, and so should be short. The second part in the curly brackets should contain the longer and more descriptive caption text.

% The \verb|\decoRule| command is optional and simply puts an aesthetic horizontal line below the image. If you do this for one image, do it for all of them.

% \LaTeX{} is capable of using images in pdf, jpg and png format.

% \subsection{Typesetting mathematics}

% If your thesis is going to contain heavy mathematical content, be sure that \LaTeX{} will make it look beautiful, even though it won't be able to solve the equations for you.

% The \enquote{Not So Short Introduction to \LaTeX} (available on \href{http://www.ctan.org/tex-archive/info/lshort/english/lshort.pdf}{CTAN}) should tell you everything you need to know for most cases of typesetting mathematics. If you need more information, a much more thorough mathematical guide is available from the AMS called, \enquote{A Short Math Guide to \LaTeX} and can be downloaded from:
% \url{ftp://ftp.ams.org/pub/tex/doc/amsmath/short-math-guide.pdf}

% There are many different \LaTeX{} symbols to remember, luckily you can find the most common symbols in \href{http://ctan.org/pkg/comprehensive}{The Comprehensive \LaTeX~Symbol List}.

% You can write an equation, which is automatically given an equation number by \LaTeX{} like this:
% \begin{verbatim}
% \begin{equation}
% E = mc^{2}
% \label{eqn:Einstein}
% \end{equation}
% \end{verbatim}

% This will produce Einstein's famous energy-matter equivalence equation:
% \begin{equation}
% E = mc^{2}
% \label{eqn:Einstein}
% \end{equation}

% All equations you write (which are not in the middle of paragraph text) are automatically given equation numbers by \LaTeX{}. If you don't want a particular equation numbered, use the unnumbered form:
% \begin{verbatim}
% \[ a^{2}=4 \]
% \end{verbatim}

% %----------------------------------------------------------------------------------------

% \section{Sectioning and Subsectioning}

% You should break your thesis up into nice, bite-sized sections and subsections. \LaTeX{} automatically builds a table of Contents by looking at all the \verb|\chapter{}|, \verb|\section{}|  and \verb|\subsection{}| commands you write in the source.

% The Table of Contents should only list the sections to three (3) levels. A \verb|chapter{}| is level zero (0). A \verb|\section{}| is level one (1) and so a \verb|\subsection{}| is level two (2). In your thesis it is likely that you will even use a \verb|subsubsection{}|, which is level three (3). The depth to which the Table of Contents is formatted is set within \file{MastersDoctoralThesis.cls}. If you need this changed, you can do it in \file{main.tex}.

% %----------------------------------------------------------------------------------------

% \section{In Closing}

% You have reached the end of this mini-guide. You can now rename or overwrite this pdf file and begin writing your own \file{Chapter1.tex} and the rest of your thesis. The easy work of setting up the structure and framework has been taken care of for you. It's now your job to fill it out!

% Good luck and have lots of fun!

% \begin{flushright}
% Guide written by ---\\
% Sunil Patel: \href{http://www.sunilpatel.co.uk}{www.sunilpatel.co.uk}\\
% Vel: \href{http://www.LaTeXTemplates.com}{LaTeXTemplates.com}
% \end{flushright}
